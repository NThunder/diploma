\fontsize{14}{16}\selectfont
\begin{abstract}
    \fontsize{14}{16}\selectfont
    \begin{center}
        \fontsize{14}{16}\selectfont
        \large{Исследование стратегий маскирования для semi-supervised предобучения моделей в задаче распознавания текста} \\
    \large\textit{Бухтуев Григорий Андреевич} \\[1 cm]
    \end{center}

    Распознавание текста на изображениях (OCR) играет важную роль в различных приложениях, от цифровизации документов до систем помощи водителю.  Semi-supervised обучение, использующее как размеченные, так и неразмеченные данные, является многообещающим подходом для повышения эффективности моделей OCR.  В данной работе исследуется потенциал различных стратегий маскирования в semi-supervised обучении для повышения точности распознавания японского текста на изображениях. 

    В работе рассматриваются три основные стратегии маскирования:  маскирование патчей входного изображения, маскирование входных представлений декодера и их комбинация. Проведен ряд экспериментов с различными архитектурами моделей.  Для оценки качества разработанных моделей использовались  метрики точности распознавания на уровне символов и фрагментов. Анализ проводился на тестовой выборке, состоящей из текстов на японском языке. 
    
    Результаты экспериментов показали, что применение стратегий маскирования позволяет повысить точность распознавания текста по сравнению с базовой моделью, обученной без маскирования. Комбинация маскирования патчей изображения и входных представлений декодера продемонстрировала наилучшие результаты.
    
    Проведенное исследование вносит вклад в развитие области semi-supervised  обучения для OCR, демонстрируя эффективность различных стратегий маскирования. Полученные результаты могут быть использованы для разработки более точных и эффективных систем распознавания японского текста, а предложенные  подходы  могут быть  адаптированы и  для  других  языков.
    \vfill

\end{abstract} 
\newpage