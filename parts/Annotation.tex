\begin{abstract}

    \begin{center}
        \large{Исследование стратегий маскирования для semi-supervised предобучения моделей в задаче распознавания текста} \\
    \large\textit{Бухтуев Григорий Андреевич} \\[1 cm]
    \end{center}

    Распознавание текста на изображениях (OCR)–  важная задача с множеством применений. В этой работе мы исследуем, как  предобучение с использованием  маскирования (masked pre-training), популярный подход в NLP и CV,  может повысить точность  OCR моделей. 

    Фокус исследования направлен на задачу распознавания японского языка на изображениях, для чего используется внутренний датасет объемом более 2 миллионов образцов.

    Вместо стандартного подхода с end-to-end обучением, мы применяем предварительное обучение модели с использованием различных  методов маскирования, включая: 
    
    Маскирование патчей входного изображения: Анализ влияния размера патчей и различных архитектур, вдохновленных Masked Autoencoders.
    
    Маскирование  входных представлений декодера: Исследование потенциала  маскирования на уровне  слоя представления.
    
    Совместное применение маскирования патчей и представлений: Анализируем эффективность  комбинации  разных уровней маскирования для достижения максимальной производительности модели.
   
    В  работе  представлены:

    Детальное описание  исследуемых  методов  и комбинаций  маскирования, а также  используемых архитектур.
    Сравнительный  анализ  эффективности  различных  подходов  на  базе  ряда  метрик.
    Анализ  влияния  размера  патчей,  сочетаний методов  и  других  гиперпараметров. 

    Наши эксперименты показывают, что:

    Предобучение с маскированием значительно улучшает точность распознавания,  превосходя  базовую модель.
    Наиболее эффективные стратегии маскируют как входные изображения, так и представления декодера.
    Мы выявили  оптимальные  конфигурации  маскирования  и  размера  патчей,  которые  позволяют  добиться  наилучших  результатов.
    
    Эта работа вносит вклад в развитие  OCR  и  предлагает новые идеи для  semi-supervised обучения моделей  распознавания текста.
    \vfill

\end{abstract}
\newpage