\begin{abstract}

    \begin{center}
        \large{Исследование стратегий маскирования для semi-supervised предобучения моделей в задаче распознавания текста} \\
    \large\textit{Бухтуев Григорий Андреевич} \\[1 cm]
    \end{center}

    Распознавание текста на изображениях (STR) играет важную роль в различных приложениях, от цифровизации документов до систем помощи водителю.  Semi-supervised обучение, использующее как размеченные, так и неразмеченные данные, является многообещающим подходом для повышения эффективности моделей STR.  В данной работе исследуется потенциал различных стратегий маскирования в semi-supervised  обучении для повышения точности распознавания японского текста на изображениях. 

    В работе рассматриваются три основные стратегии маскирования:  маскирование патчей входного изображения, маскирование входных представлений декодера и их комбинация. Проведен ряд экспериментов с различными архитектурами моделей, размерами патчей и значениями mask ratio.  Для оценки качества разработанных моделей использовалась метрика  Character Accuracy (CharAcc) на тестовом наборе данных с японским текстом. 
    
    Результаты экспериментов показали, что все три рассмотренные стратегии маскирования превосходят baseline модель, обученную без использования маскирования.  Наилучшие результаты были достигнуты при комбинировании маскирования патчей изображения и входных представлений декодера, что позволило достичь прироста CharAcc на X \% по сравнению с baseline.  
    
    Данная работа вносит вклад в развитие области semi-supervised  обучения для STR, демонстрируя эффективность различных стратегий маскирования. Полученные результаты могут быть использованы для разработки более точных и эффективных систем распознавания японского текста, а предложенные  подходы  могут быть  адаптированы и  для  других  языков.
    \vfill

\end{abstract}
\newpage