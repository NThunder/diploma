\section{Постановка задачи}
\label{sec:Chapter1} \index{Chapter1}

Есть целое множество статей с различными стратегиями применения маскирования в предобучении задач OCR, STR.
Цель исследовать эффективность  различных стратегий маскирования  в  semi-supervised обучении для повышения точности распознавания текста на изображениях (STR)  и определить оптимальный подход для  данной задачи.

Задачи:

1. Реализовать и сравнить  различные  стратегии  маскирования  входных данных:
    Маскирование патчей входного изображения.
    Маскирование входных представлений декодера.
    Комбинирование  маскирования  патчей  изображения  и  входных  представлений  декодера.
2. Провести  эксперименты  с  разными  архитектурами  моделей  и  параметрами  маскирования:
    Использовать  разные  размеры  патчей  изображения.
    Варьировать  mask ratio  (процент  замаскированных  данных).
    Исследовать  влияние  разных  функций  потерь  на  этапе  предобучения.
3. Оценить  качество  полученных  моделей  по  метрике  CharAcc  на  задаче  распознавания  текста  на  японском языке.
4. Проанализировать  полученные  результаты  и  сделать  выводы  об  эффективности  разных  стратегий  маскирования  и  их  влиянии  на  точность  распознавания  текста.

Критерии оценки решения:

Точность  распознавания  текста (CharAcc):  чем  выше  значение  метрики  на  тестовом  наборе  данных,  тем  лучше.
Масштабируемость  решения:  возможность  эффективно  обучать  и  использовать  модель  на  больших  наборах  данных.
Интерпретируемость  результатов:  возможность  проанализировать  и  объяснить,  почему  одна  стратегия  маскирования  оказалась  эффективнее  другой.

Описание  данных:

Датасет  с  изображениями,  содержащими  печатный  текст  на  японском  языке.
Размеченные  данные  для  обучения  и  валидации  моделей.

Ожидаемые  результаты:

Разработка  эффективной  стратегии  маскирования  для  semi-supervised  обучения  моделей  STR.
Создание  модели  STR,  превосходящей  baseline  модель  по  точности  распознавания  текста.
Анализ  влияния  разных  параметров  маскирования  на  качество  полученных  моделей.
Формулировка  рекомендаций  по  использованию  маскирования  в  semi-supervised  обучении  для  задачи  STR.



\newpage
