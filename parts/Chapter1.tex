\section{Постановка задачи}
\label{sec:Chapter1} \index{Chapter1}


\subsection{Актуальность} 

Распознавание текста на изображениях (OCR) является важной задачей компьютерного зрения с широким спектром практических применений. Semi-supervised learning представляет собой перспективный подход к повышению эффективности OCR-моделей, позволяя использовать как ограниченные размеченные, так и обширные неразмеченные данные. В частности,  предобучение с использованием маскирования демонстрирует высокую эффективность в задачах  самообучения  моделей компьютерного зрения.

\subsection{Проблема}

Несмотря на многообещающие результаты,  оптимальная стратегия  маскирования  для semi-supervised  предобучения OCR-моделей остается открытым вопросом. Существуют  различные  подходы  к маскированию:  на  уровне  входных  данных  (патчи изображения),  на уровне скрытых  представлений  (признаки) и их комбинации.   Выбор наиболее эффективной стратегии  зависит от  множества  факторов,  включая  архитектуру  модели,  характер  данных  и  требования  к  точности распознавания.

\subsection{Цель работы} 

Провести комплексное сравнение различных  стратегий  маскирования  для semi-supervised  предобучения  моделей  OCR  и определить оптимальный подход для повышения точности распознавания текста на изображениях на японском языке.

\subsection{Задачи исследования}

\begin{enumerate}
  \item Проанализировать существующие  стратегии  маскирования  для предобучения моделей  OCR.
  \item Разработать  и  реализовать  модификации  архитектуры  OCR-модели,  поддерживающие  различные  стратегии  маскирования.
  \item Провести  экспериментальное  исследование  эффективности  различных  стратегий  маскирования  на  задаче  распознавания  японского текста.
  \item Выбрать  и  обосновать  оптимальную  стратегию  маскирования  для semi-supervised  предобучения  OCR-моделей  на  японском  языке. 
\end{enumerate}

\subsection{Практическая значимость} 

Результаты  исследования  позволят  разработать  более  эффективные  методы  обучения  OCR-моделей,  что  актуально  для  широкого  спектра  приложений,  связанных  с  обработкой  изображений  и  информации,  содержащей  текст.

\newpage
