\section*{Приложение}
\addcontentsline{toc}{section}{Приложение}
\label{sec:Apendix} \index{Apendix}

\subsection{Дополнительные исследования}

\begin{itemize}
    \item \textbf{Уменьшение размера патчей}

    В ходе предварительных экспериментов было замечено, что использование патчей размером 16x16 пикселей может приводить к тому, что некоторые символы на изображении полностью перекрываются одним маскированным патчем. Это может негативно сказаться на качестве обучения модели, поскольку информация о таких символах будет потеряна. 

    Для решения данной проблемы был исследован подход с использованием патчей меньшего размера - 8x8 пикселей. Предварительные эксперименты показали, что уменьшение размера патчей позволяет снизить вероятность полного перекрытия символов и улучшить качество реконструкции изображения на этапе предварительного обучения. 
    \begin{figure}[H]
        \includegraphics[scale=0.9]{8px.png}
        \caption{Пример использования патчей меньшего размера при обучении архитектуры 1. (a) Оригинальное изображение. (b) Изображение с замаскированными патчами. (c) Реконструированное изображение. (d) Совмещенное изображение, показывающее оригинальные и реконструированные патчи. Параметры: размер патча – 8x8 пикселей, mask ratio – 75\%,  Эпох предварительного обучения – 60.}
    \end{figure}
    Однако, из-за ограничений по времени и вычислительным ресурсам, провести полное исследование эффективности данного подхода, включая этап тонкой настройки (finetune) модели для задачи распознавания текста, не удалось.  
\end{itemize}