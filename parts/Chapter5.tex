\section{Заключение}
\label{sec:Chapter5} \index{Chapter5}

В данной главе представлены результаты экспериментов по исследованию эффективности различных стратегий маскирования в semi-supervised обучении для повышения точности распознавания японского текста на изображениях.  

 4.1 Результаты  маскирования патчей изображения 

Архитектура 1.1 (Mask ratio: 25процента, patch size: 16px):CharAcc на тестовом наборе данных составила 96.14процента, что  превышает  baseline результат (95.8 процента) на 0.34 процента. 
Архитектура 1.2 (Mask ratio: 75процента, patch size: 16px):Увеличение mask ratio  до 75 процента  привело к дополнительному  улучшению  показателя  CharAcc  до 96.32 процента, что  составляет  прирост  в 0.52 процента относительно  baseline.
Архитектура 2 (SimMIM, Mask ratio: 75процента, patch size: 16px): Использование  архитектуры  SimMIM  с  mask ratio 75 процента  показало  схожий  результат  с  Архитектурой  1.2 —  96.29 процента  CharAcc,  что  также  выше  baseline.

Выводы:

Маскирование патчей изображения  позволяет  улучшить  точность  распознавания  текста по  сравнению  с  обучением  без  маскирования.
Более  высокий  mask ratio  приводит  к  более  значительному  повышению  точности. 
Разные  архитектуры  моделей  с  маскированием  патчей  могут  демонстрировать  сравнимые  результаты.

 4.2 Результаты  маскирования  входных  представлений  декодера

Архитектура 4 (Mask ratio: 75процента):Применение маскирования  к  входным  представлениям  декодера  с  mask ratio 75 процента  привело к  значительному  росту  показателя  CharAcc  до 96.4 процента,  что  составляет  0.6 процента улучшения  относительно  baseline.

Выводы:

Маскирование  входных  представлений  декодера  также  эффективно  для  повышения  точности  распознавания  текста.
Данный  подход  показал  себя  более  эффективным,  чем  маскирование  патчей  изображения.

 4.3 Результаты  комбинированного  маскирования

Архитектура 3 (Masked Vision-Language Transformer, Mask ratio: 25процента): Использование  архитектуры,  совмещающей  в  себе  механизмы  маскирования  патчей  изображения  и  входных  представлений  декодера,  привело  к  CharAcc  на  уровне  96.34 процента.
Архитектура 5 (Mask ratio: 33 процента + 33процента): Комбинация  маскирования  патчей  изображения  с  mask ratio 33 процента  и  маскирования  входных  представлений  декодера  с  тем  же  mask ratio  показала  наилучший  результат—  96.5 процента  CharAcc.  Это  составляет  прирост  в  0.7 процента относительно  baseline.

Выводы:

Комбинация  разных  стратегий  маскирования  позволяет  добиться  наибольшей  точности  распознавания  текста. 
Подбор  оптимальных  параметров  маскирования  для  каждого  из  подходов  играет  важную  роль.

 4.4  Степень решения задачи

Полученные  результаты  свидетельствуют  о  том,  что  поставленная  задача  решена.  Удалось  разработать  и  исследовать  различные  стратегии  маскирования  для  semi-supervised  обучения  моделей  STR  на  японском  языке.  Все  исследованные  стратегии  маскирования  превзошли  baseline  модель  по  точности  распознавания  текста.  Наилучший  результат  был  достигнут  при  комбинировании  маскирования  патчей  изображения  и  входных  представлений  декодера  с  определенным  соотношением  mask ratio. 
\newpage