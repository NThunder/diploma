\section{Описание практической части}
\label{sec:Chapter4} \index{Chapter4}

Описание кода

 1. Выбор языка и библиотек

Язык программирования:Python 
Библиотеки:
    PyTorch: для создания и обучения нейронных сетей, работы с тензорами, автоматического дифференцирования.
    OpenCV (cv2):для предобработки изображений (загрузка, изменение размера, нормализация).
    NumPy:для работы с массивами и математическими операциями.
    tqdm:для визуализации прогресса обучения.

Мотивы выбора:

Python: популярный язык  с  большим  сообществом,  особенно  в  области  машинного  обучения.  Обладает  простым  синтаксисом  и  множеством  библиотек. 
PyTorch: гибкий  и  мощный  фреймворк,  позволяющий  легко  создавать  и  обучать  сложные  нейронные  сети.

 2. Архитектура кода

Код  организован  в  виде  модульной  структуры,  которая  включает  в  себя  следующие  компоненты:

data  loader.py: Загрузка  и  предобработка  датасета,  реализация  маскирования  патчей  изображения.
models.py: Определение  архитектуры  модели  STR,  включая  энкодер,  декодер  и  механизмы  маскирования.
train.py: Функции  для  обучения  и  валидации  модели,  сохранения  весов  и  ведения  логов.
evaluate.py: Оценка  обученной  модели  на  тестовом  наборе  данных  и  расчет  метрик  качества.
utils.py: Вспомогательные  функции,  например,  для  работы  с  конфигурационными  файлами,  визуализации  результатов.

 3. Схема функционирования

1. Загрузка  и  предобработка  данных: Датасет  загружается,  изображения  предобрабатываются  (изменение  размера,  нормализация),  текстовые  метки  преобразуются  в  подходящий  формат. 
2. Создание  и  обучение  модели: Создается  экземпляр  модели  STR  с  выбранной  архитектурой  и  механизмами  маскирования.  Модель  обучается  на  обучающем  наборе  данных  с  использованием  заданных  параметров  (оптимизатор,  функция  потерь,  количество  эпох). 
3. Оценка  модели: Обученная  модель  оценивается  на  тестовом  наборе  данных  для  расчета  метрик  точности  распознавания  текста  (CharAcc). 

 4. Теоретическая сложность алгоритма

Теоретическая сложность алгоритма  обучения  нейронной  сети  зависит  от  многих  факторов,  таких  как  архитектура  сети,  размер  датасета,  выбранный  оптимизатор  и  другие  параметры.  В  общем  случае,  обучение  нейронной  сети  —  задача  NP-трудная.  

 5. Характеристики  функционирования

Скорость: Скорость  работы  кода  зависит  от  вычислительной  мощности  оборудования  (CPU,  GPU),  размера  модели  и  датасета,  а  также  от  эффективности  реализации. 
Память: Объем  используемой  памяти  зависит  от  размера  модели,  размера  батча  и  разрешения  изображений. 

\newpage
